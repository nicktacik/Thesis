

\chapter{Initial Data for Spinning Neutron Stars in Mixed Binaries}

\section{Introduction}
The spectacular detection of merging binary black holes by Advanced
LIGO (\cite{PhysRevLett.116.241103},~\cite{Abbott:2016blz}) has dawned the beginning of the era of gravitational wave astronomy. Black-hole - neutron star (Bh-Ns) binaries are now the only compact object binary whose existence has not been direclty confirmed. They remain, however, an important potential source of gravitational waves for advanced ground-based detectors, with an expected event rate of approximately ten per year(\cite{AbadieLSC:2010}), albeit with a large uncertainity. In addition to gravitational waves, Bh-Ns mergers can be an exciting source of electromagnetic radiation and give further clues to the violent processes that occur during the merger. If a massive disk is left from the merger, it could lead to a short-duration gamma ray burst (SGRB) and material ejected during the merger could radiate a signal such as a "kilonova"(\cite{metzger:11}).

Numerical relativity simulations are important to accurately study
both the gravitational waves and electromagnetic emission produced in
Bh-Ns mergers.  The parameter space for Bh-Ns binary simulations is
relatively large. The mass ratio, $q$, and black hole spin, $\vec{\chi}$, have been of particular interest in numerical simulations. \red{cites}
%Neutron star equation of state has also been of interest in more recent simulations that can incorporate more microphysics. 
%\cite{Foucart2012} gave a semi-analytic description of how this
%combination (of mass ratio and spin magnitude) can predict when
%neutron star disruption occurs. 
The neutron star equation of state is also of
great interest, especially in more recent simulations that can
incorporate more microphysics. \red{cites} The size of the star is directly
linked to when tidal disruption occurs. One aspect that has not been
studied, however, is the effect of neutron star spin, as all
simulations to date use irrotational neutron stars in their Bh-Ns
binaries. For Ns-Ns binaries, in constrast, a significant number of
studies, investigate spinning neutron stars in Ns-Ns binaries
(\cite{Baumgarte:2009fw},~\cite{Tichy:2011gw},~\cite{East:2012zn},~\cite{Tichy:2012rp},~\cite{Bernuzzi:2013rza},~\cite{Kastaun:2013mv},~\cite{Tsatsin:2013jca},~\cite{Dietrich:2015pxa},~\cite{East:2015yea},~\cite{Tsokaros:2015fea},~\cite{Tacik:2015tja}). Since
no Bh-Ns binaries have been directly observed, the Ns spins are, at
least observationally, unconstrained. A spinning neutron star will affect the gravitational waveforms and cause the inspiral to proceed more slowly (for spin-aligned Ns). The spin can be important for gravitational wave detection and can cause appreciable mismatch with non-spinning templates, especially at lower Bh-Ns mass ratios (\cite{Ajith:2011ec}). The spin could also affect the time of Ns disruption, as the stellar material will be less tightly bound to the stellar surface.

Recently, \cite{Tacik:2015tja} used the {\tt SpEC} code to create and
evolve initial data sets of binary neutron star systems with arbitrary
spins. In this chapter, we extend this code to create initial data for
Bh-Ns systems where the neutron star spin is arbitrary. The structure of this
chapter is as follows: In section~\ref{sec:Bh-NSIDF}, we review the standard numerical relativity initial data formalism, as well as the formalism
developed in \cite{Tichy:2011gw} to create binaries with spinning Ns,
and discuss how this is extended to Bh-Ns systems. In section~\ref{sec:BhNSNumMethods} we discuss the numerical methods used by our initial data solver.
In section~\ref{sec:BhNSResults}, we create a number of initial data sets to demonstrate the robustness of our solvers. We extend the results of~\cite{Foucart:2013a} to incorporate different neutron star spins,
and we also create a variety of initial data sets with various values
of neutron star spin, black hole spin, and mass ratio, to demonstrate
robustness across the Bh-Ns parameter space. We demonstrate initial
data convergence and investigate properties of the solution. Finally,
we conclude the chapter in section~\ref{sec:Conc}.
In this chapter we use units where $G=c=M_{\odot}=1$.

\section{Initial Data Formalism}
\red{Consider shortening this section and instead reference previous sections.}
\label{sec:Bh-NSIDF}
In this section we will discuss the formalism used to solve the
Einstein field equations and create quasi-equilibrium initial data for
Bh-Ns binaries with spinning neutron
stars. We begin with the $3+1$ decomposition of the space-time metric
tensor,
\begin{equation}
g_{\mu\nu}dx^{\mu}dx^{\nu} = -\alpha^2dt^2 + \gamma_{ij}\left(dx^i +
  \beta^idt\right)\left(dx^j+\beta^jdt\right),
\end{equation}
where $\alpha$ is the lapse function, $\beta^i$ is the shift vector,
and $\gamma_{ij}$ is the induced metric on a spatial hypersurface
$\Sigma(t)$. The normal vector $n^{\mu}$ to $\Sigma(t)$ is related to
the coordinate time $t$ by
\begin{equation}
t^{\mu} = \alpha n^{\mu} + \beta^{\mu}.
\end{equation}
The extrinsic curvature of $\Sigma(t)$ is given by
\begin{equation}
K_{\mu\nu} = -\frac{1}{2}\mathcal{L}_n\gamma_{\mu\nu},
\end{equation}
where $\gamma_{\mu\nu}=g_{\mu\nu}+n_{\mu}n_{\nu}.$ and $\mathcal{L}_n$
is the Lie derivative in the direction of the $n^{\mu}$. We restrict our
attention to the spatial part of the extrinsic curvature, $K^{ij}$,
since $n_{\mu}K^{\mu\nu}=0$ by construction. It is convenient to
decompose it into its trace and trace-free parts,
\begin{equation}
K^{ij} = A^{ij}+\frac{1}{3}K\gamma_{ij}.
\end{equation}
The matter in the system is modelled with the stress-energy tensor of
a perfect fluid 
\begin{equation}
T_{\mu\nu}=\left(\rho+P\right)u_{\mu}u_{\nu}+Pg_{\mu\nu},
\end{equation}
where $\rho$ is the fluid's energy density, $P$ is its pressure, and
$u^{\mu}$ is its four-velocity. It is further useful to define the
projections of the matter quantities,
\begin{eqnarray}
E &=& T^{\mu\nu}n_{\mu}n_{\nu},\\
S &=& \gamma^{ij}\gamma_{i\mu}\gamma_{j\nu}T^{\mu\nu}, \\
J^i &=& -\gamma^{i}_{\mu}T^{\mu\nu}n_{\nu}.
\end{eqnarray}
The spatial metric is decomposed with the conformal transformation
\begin{equation}
\gamma_{ij}=\Psi^4\tilde{\gamma}_{ij},
\end{equation}
where $\Psi$ is called the conformal factor, and $\tilde{\gamma}_{ij}$
is the conformal metric. The other quantities in the initial value
problem use the following conformal transformations:
\begin{eqnarray}
E &=& \Psi^{-6}\tilde{E}, \\
S &=& \Psi^{-6}\tilde{S}, \\
J^i &=& \Psi^{-6}\tilde{J}^i, \\
A^{ij} &=& \Psi^{-10}\tilde{A}^{ij}, \\
\alpha &=& \Psi^{6}\tilde{\alpha}. 
\end{eqnarray}
$\tilde{A}^{ij}$ is related to the shift and to the time derivative of
the conformal metric, $\tilde{u}_{ij}=\partial_t\tilde{\gamma}_{ij}$,
by
\begin{equation}
\tilde{A}^{ij} =
\frac{1}{2\tilde{\alpha}}\left[\left(\tilde{\mathrm{L}}\beta\right)^{ij}-\tilde{u}^{ij}\right],
\end{equation}
where $\tilde{L}$ is the conformal longitudinal operator,
\begin{equation}
\left(\tilde{L}V\right)^{ij}=\tilde{\nabla}^iV^j + \tilde{\nabla}^jV^i
- \frac{2}{3}\tilde{\gamma}^{ij}\tilde{\nabla}_kV^k.
\end{equation}
We solve the extended conformal thin sandwich (XCTS) equations, which are a set of 5
coupled non-linear equations.
\begin{equation}
\label{eq:XCTS-Shift}
2\tilde{\alpha}\bigg[\tilde{\nabla}_j\left(\frac{1}{2\tilde{\alpha}}\big(\tilde{L}\beta\big)^{ij}\right)-\tilde{\nabla}_j\left(\frac{1}{2\tilde{\alpha}}\tilde{u}^{ij}\right)
-\frac{2}{3}\Psi^6\tilde{\nabla}^iK-8\pi\Psi^4\tilde{J}^i\bigg] =0,
\end{equation}

\begin{equation}
\label{eq:XCTS-ConformalFactor}
\tilde{\nabla}^2\Psi - \frac{1}{8}\Psi\tilde{R} -
\frac{1}{12}\Psi^5K^2 
+\frac{1}{8}\Psi^{-7}\tilde{A}_{ij}\tilde{A}^{ij} +
2\pi\Psi^{-1}\tilde{E}=0
\end{equation}

\begin{equation}
\label{eq:XCTS-Lapse}
\tilde{\nabla}^2\left(\tilde{\alpha}\Psi^7\right) -
\left(\tilde{\alpha}\Psi^7\right)\bigg[\frac{1}{8}\tilde{R}+\frac{5}{12}\Psi^4K^2+\frac{7}{8}\Psi^{-8}\tilde{A}_{ij}\tilde{A}^{ij}
+2\pi\Psi^{-2}\big(\tilde{E}+2\tilde{S}\big)\bigg]=-\Psi^5\left(\partial_{t}K
- \beta^{k}\partial_kK\right).
\end{equation}
These are solved
for the conformal factor, $\Psi$, the densitized lapse, $\alpha\Psi$,
and the shift, $\beta^i$.

The matter content of $\Sigma(t)$ is determined by $\tilde{E}$,
$\tilde{S}$, and $\tilde{J}^i$. The free data are $\gamma_{ij}$,
$\tilde{u}_{ij}$, $K$ and $\partial_t K$. $\tilde{u}_{ij}=0$ and
$\partial_t K=0$ are natural choices in a coordinate system corotating
with the binary. The
choice of the conformal metric will be discussed in section~\ref{sec:BhNSNumMethods}.

Let us now further discuss the matter content of the neutron star. The
energy density of the fluid is $\rho=\rho_0\left(1+\epsilon\right)$,
where $\rho_0$ is the baryon density and $\epsilon$ is the internal
energy. The specific enthalpy of the fluid is
\begin{equation}
h=1+\epsilon+\frac{P}{\rho_0}.
\end{equation}
It is convenient to introduce the following projections of the four
velocity $u^{\mu}$ and the three-velocity $U^{\mu}$:
\begin{eqnarray}
u^{\mu} &=& \gamma_n\left(n^u+U^\mu\right), \\
U^{\mu}n_{\mu}&=&0,\\
U^i_0 &=& \frac{\beta^i}{\alpha}, \\
\gamma_0 &=& \left(1 - \gamma_{ij}U^i_0U^j_0\right)^{-1/2}, \\
\gamma_n &=& \left(1 - \gamma_{ij}U^iU^j\right)^{-1/2}, \\
\gamma &=& \gamma_n\gamma_0\left(1-\gamma_{ij}U^iU^j_0\right).
\end{eqnarray}
Following \cite{Tichy:2011gw}, the three-velocity is written as the sum
of an irrotational part (the gradient of a potential $\phi$) and a
rotational part $W^i$,
\begin{equation}
U^i =
\frac{\Psi^{-4}\tilde{\gamma}^{ij}}{h\gamma_n}\left(\partial_j\phi+W_j\right).
\end{equation}
$W^i$ is a freely chosen, divergence-free vector field in this
formalism; we will discuss the choice of $W^i$ in section~\ref{sec:BhNSNumMethods}.

The matter fluid must satisfy the continuity equation and the Euler
equation. 
Under the assumptions made in \cite{Tichy:2011gw}, the continuity equation is a second order elliptic equation for the potential $\phi$:
\begin{equation}
\label{eq:Continuity1}
\frac{\rho_0}{h}\nabla^{\mu}\nabla_{\mu}\phi+\left(\nabla^{\mu}\phi\right)\nabla_{\mu}\frac{\rho_0}{h}=0.
\end{equation}
This can be re-written as
\begin{eqnarray}
\label{eq:Continuity2}
\rho_0\,\bigg\{\!\!-\tilde{\gamma}^{ij}\partial_i\big(\partial_j\phi+W_j\big)  &+& \frac{h\beta^i\Psi^4}{\alpha}\partial_i\gamma_n + hK\gamma_n\Psi^4+\Big[\tilde{\gamma}^{ij}\tilde{\Gamma}^k_{ij}+\gamma^{ik}\partial_i\big(\ln \frac{h}{\alpha\Psi^2}\big)\Big] 
\big(\partial_k\phi+W_k\big) \bigg\} \nonumber\\
&=&\tilde{\gamma}^{ij}\big(\partial_i\phi+W_i\big)\partial_j\rho_0 - \frac{h\gamma_n\beta^i\Psi^4}{\alpha}\partial_i\rho_0.
\label{eq:Continuity}
\end{eqnarray}
The Euler equation is solved for the specific enthalpy $h$. The
solution is, as shown in \cite{Tichy:2011gw}:
\begin{equation}
h = \sqrt{L^2 -
  \left(\nabla_i\phi+W_i\right)\left(\nabla^i\phi+W^i\right)},
\end{equation}
where
\begin{equation}
L^2 =
\frac{b+\sqrt{b^2-4\alpha^4\left(\left(\nabla_i\phi+W_i\right)W^i\right)^2}}{2\alpha^2},
\end{equation}
and
\begin{equation}
b =
\left(\beta^i\nabla_i\phi+C\right)^2+2\alpha^2\left(\nabla_i\phi+W_i\right)W^i.
\end{equation}

The boundary condition on $\phi$ at the surface of the neutron star can deduced from the $\rho_0\rightarrow 0$ limit of the continuity equation:
\begin{equation}
\tilde{\gamma}^{ij}\left(\partial_i\phi+W_i\right)\partial_j\rho_0=\frac{h\gamma_n\beta^i\Psi^4}{\alpha}\partial_i\rho_0.
\end{equation}
Note that $\phi$ is only solved for inside the neutron stars, while
the metric variables are solved for everywhere.
The boundary condition at infinity (which is in practice, in our
computational grid, at $R=10^{10}$) are the requirement of a Minkowski
metric in the inertial frame (\cite{FoucartEtAl:2008}):
\begin{eqnarray}
\beta_0&=&0,\\
\alpha\Psi &=& 1,\\
\Psi &=&1.
\end{eqnarray}
The interior of the black hole is excised from the computation
domain. The boundary conditions at the surface of the black hole
apparent horizon, $\mathcal{H}$, are~(\cite{Cook2004}):
\begin{align}
\tilde{s}^k\nabla_k\log\Psi=&-\frac{1}{4}\left(\tilde{h}^{ij}\tilde{\nabla}^i\tilde{s}_j-\Psi^2h^{ij}K_{ij}\right)
\qquad &{\rm on} ~\mathcal{H} \label{eq:BhBoundary}\\
\beta_{\perp}=&\alpha &{\rm on} ~\mathcal{H} \label{eq:BhBoundary2} \\
\beta_{\parallel}=&\Omega_{j}^{BH}x_k\epsilon^{ijk}
&{\rm on}
~\mathcal{H} \label{eq:BhBoundary3}
\end{align}
where $s^i=\Psi^{-2}\tilde{s}^i$ is the outward pointing unit normal to the apparent horizon
surface, $h^{ij}$ is the 2-metric on the surface, $x_i$ are the
Cartesian coordinates relative to the centre of the black hole and $\Omega_j^{BH}$ is a
free vector that determines the spin of the black hole.
The force balance equation at the centre of the neutron star, $c^i$ is 
\begin{equation}
\nabla\log h=0 \qquad {\rm at}~x^i=c^i.
\end{equation}
We can re-write this equation as (\cite{Tichy:2011gw}) 
 \begin{equation}
\label{eq:OmegaDriver}
\nabla\ln\left(\alpha^2-\gamma_{ij}\beta^{i}\beta^{j}\right)=-2\nabla\ln\Gamma,
\end{equation}
where
\begin{equation}
\Gamma
=\frac{\gamma_n\left(1-\left(\beta^i+\frac{W^i\alpha}{h\gamma_n}\right)\frac{\nabla_i\phi}{\alpha
    h\gamma_n}- \frac{W_i W^i}{\alpha^2\gamma_n^2}\right) } { \sqrt{ 1
    - \left(\frac{\beta^i}{\alpha}+\frac{W^i}{h\gamma_n}\right)
    \left(\frac{\beta_i}{\alpha}+\frac{W_i}{h\gamma_n}\right) } } ,
\end{equation}
which is a second order equation for the orbital
frequency $\Omega$, since $\beta^i=\beta^i_0 +
\vec{\Omega}\times\vec{r} + \dot{a}\vec{r}$, where $\beta^I_0$ is the
shift in the inertial frame. If desired, this equation can be solved to find a
best guess for the orbital frequency. Alternatively, eccentricity
removal techniques, such as those used in \cite{Tacik:2015tja} can be
used to find the best value of the orbital frequency.

$W^i$ is chosen as as to give the Ns a uniform rotational
profile. Following our work in \cite{Tacik:2015tja}, we use
\begin{equation}
\label{eq:RotationTerm}
W^i=\epsilon^{ijk}\omega^jr^k,
\end{equation}
where $\epsilon^{ijk}=\{\pm1,0\}$, $r^k$ is the position vector relative to the centre of the star, and $\omega^j$ is a freely chosen constant vector.

The angular momentum of the black hole is computed as  (\cite{FoucartEtAl:2008}):
\begin{equation}
\label{eq:BhSpin}
S=\frac{1}{8\pi}\oint_{\mathcal{H}}\phi^is^jK_{ij}dA
.
\end{equation}
 In a space-time with aziumuthal symmetry, $\phi^i$ would represent
 the exact aziumuthal Killing vector field generated by this symmetry.
Since aziumuthal symmetry is not present in a binary system, we
instead use an {\it approximate} Killing vector. It is computed by solving a
shear minimization eigenvalue problem - see
\cite{Cook2007,Lovelace2008}  for details. The dimensionless spin is defined as
\begin{equation}
\label{eq:ChiDef}
\chi=\frac{S}{M^2}
\end{equation}
where $M$ is the Christodoulou mass
\begin{equation}
M^2=M_{\rm irr}^2+\frac{S^2}{4M_{\rm irr}^2}.
\end{equation}
The irreducible mass is related to the surface area of the horizon
\begin{equation}
M_{\rm irr}=\sqrt{A/16\pi}.
\end{equation}
The same method is used to compute the dimensionless spin of the
neutron star. In particular we use Eq.~\ref{eq:BhSpin}, with
$\mathcal{H}$ replaced by the neutron star's surface, as defined in
Eq.~\ref{eq:NSSurf} to compute the star's angular momentum. Its
Arnowitt-Deser-Misner (ADM) mass is used as the normalization to
compute its dimensionless spin, analogous to the Christodoulou mass in
Eq.~\ref{eq:ChiDef}. In particular, we use the ADM mass of
an analogous rotating star with the same angular momentum in
isolation, since the ADM mass of the star in a binary in not directly
measurable. \cite{Tacik:2015tja} showed that this method of computing
neutron star spin was robust and accurate.

\section{Numerical Methods}
\label{sec:BhNSNumMethods}
The XCTS equations
(\cref{eq:XCTS-Shift,eq:XCTS-ConformalFactor,eq:XCTS-Lapse} and the
continuity equation (Eq.~\ref{eq:Continuity2}) form a set
of six non-linear coupled elliptic equations that we must now solve. We use the
pseduo-spectral multi-grid elliptic solver developed in
\cite{Pfeiffer2003} and enhanced to
incorporate matter in \cite{FoucartEtAl:2008}. The computational domain is divided into a number
of subdomains. A small cube is placed at the centre of the neutron star. Overlapping this cube is a spherical shell whose outer boundary deforms to fit the stellar surface. These are surrounded by an additional spherical shell. The black hole is represented by two concentric spherical shells. Their inner bounday is required to be an apparent horizon. Three rectangular parallelpipeds surround the axis passing through the centres of the Bh and the Ns - one between them and one on each side of the objects. An additional eight cylindrical shells are placed around the same axis to cover the intermediate field region. The far-field region is covered by a large spherical shell whose outer boundary is placed at $R=10^{10}$ using an inverse radial mapping. This domain is visualized in Fig.~\ref{fig:BhNsDomain}.
\begin{center}
\begin{figure}
\includegraphics[scale=0.45]{chap4/BhNsDomain.png}
\caption[Visualization of the Bh-Ns domain
decomposition.]{Visualization of the Bh-Ns domain decomposition. The
  black object on the left is the black hole and the orange object on
  the right is the neutron star. The blue wireframes represent the
  various cylinders and rectangular parallelpipeds in the
  domain. \red{[Make this look nicer]}}
\label{fig:BhNsDomain}
\end{figure}
\end{center}
All variables (metric and hydrodynamical) are decomposed on sets of
basis functions on each subdomain. The type of basis function used
depends on the topology on the subdomain. Finite difference schemes are
needed for hydrodynamical quantities during evolutions so as to
capture shocks, but for initial data, where shocks are not present,
spectral methods are suitable and exponentical convergence can be
achieved. The resolution of each domain is synonymous with the
number of colocation points used. They are chosen manually at the
start of the initial data solve, and then modified several times using
an adaptive scheme (see step~\ref{it:final}).  To discuss the resolution of the computational domain for the purpose of convergence tests, we will use the notation
\begin{equation}
N^{1/3}=\left(\sum_{i\in{\rm Subdomains}} N_i\right)^{1/3},
\end{equation}
where we are summing the colocation points of all of the
subdomains. This gives an indication of the average resolution per
dimension. A typical initial data solve starts with, $N^{1/3}\sim 33$ and end with $N^{1/3}\sim 80$.

Construction of initial data begins by choosing values for the
physical parameters in the problem we aim to achieve. In particular,
these are: parameters of the black hole,
\begin{itemize}
\item The black hole mass, $M_{\rm BH}$,
\item The black hole's dimensionless spin vector,
  $\vec{\chi}_{\rm BH}$,
\end{itemize}
parameters of the neutron star,
\begin{itemize}
\item The neutron star's baryon mass, $M_{b}$,
\item The neutron star's equation of state, 
\item The neutron star's spin vector, $\omega^i$,
\end{itemize}
and orbital paramters,
\begin{itemize}
\item The separation between the centres of the Bh and Ns, $D$,
\item The orbital angular velocity, $\Omega_0$,
\item The initial infall velocity parameter, $\dot{a}$
.
\end{itemize}
Additionally,  a perscription is required for the free
metric variables, $\tilde{g}_{ij}$ and $K$. We use the following: near the black hole they
are equal to the values for an isolated black hole with the same spin and boost
velocity,
written in Kerr-Schild coordinates. Near the neutron star, they are a
conformally flat metric and $K=0$. They fall-off from the black hole
to the neutron star with a function of the form $e^{(-r/w)^4}$, where
$w$ is half the coordinate separation between them.

Let's now discuss the algorithm we use to solve for the initial data. 

\begin{enumerate}
\item 
\label{it:1}
If this is the first resolution (i.e. step~\ref{it:secondlast} has not been
  completed once yet), set $\omega^i=0$, otherwise set
  to the desired value. This has been found to help with overall
  convergence, especially for high neutron star spins. Additionally, impose a maximum radius out to which to
  apply $W^i=\epsilon^{ijk}\omega^jr^k$, otherwise low density
  material at high radius can lead to spurious large velocities,
  especially when the neutron star spin is large, or when the black
  hole mass is much larger than the neutron star mass.

\item 
\label{it:solve}
Solve the non-linear XCTS equations for the metric variables (\cref{eq:XCTS-Shift,eq:XCTS-ConformalFactor,eq:XCTS-Lapse} )
  $X=\left(\beta^i,\Psi,\alpha\Psi\right)$ assuming the matter source
  terms are fixed. Update the metric variables using a relaxation
  scheme
\begin{equation}
\label{eq:Relaxation}
X^{n+1}=\lambda X^{*} + (1-\lambda)X^n,
\end{equation}
where $X^{*}$ is the result found by solving the XCTS equations, and $X^{n}$ is
the previous value of $X$.
We use $\lambda=0.3$.

\item If both the Ns and Bh have either aligned spin or zero spin,
  impose equatorial symmetry. This will be speed convergence and
  decrease computational cost.

\item
\label{it:toplevelparamsolve}
 If already at a sufficiently high resolution, skip to step~\ref{it:omega}, otherwise proceed to step~\ref{it:surface}. We generally find that after completed the first four resolutions, 
the stellar and black hole parameters are already computed to a sufficient accuracy, so therefore those steps can be skipped to decrease computational cost.

\item 
\label{it:surface}
Locate the surface of the star. The surface of the star is
  represented in terms of spherical harmonics
\begin{equation}
\label{eq:NSSurf}
R(\theta,\phi)=\sum_{m,l}^{m_{\rm max},l_{\rm max}} c_{lm}Y^{lm}(\theta,\phi)
.
\end{equation}
The coefficients $c_{lm}$ are determined by solving the relation $h\left(R\left(\theta,\phi\right)\right)=1$.
We generally use $l_{\rm max}=11$.

\item Compute the ADM linear momentum $P_{\rm ADM}$, as 
\begin{equation}
P_{\rm ADM}^i = \frac{1}{8\pi}\oint_{S_{\infty}}K^{ij}dS_j.
\end{equation}
If its norm has changed by less than 10\% in the last iteration, move
the centre of the BH so as to zero-out the momentum of the
system. In particular, find $\delta c$ satisfying $\vec{\delta c}
\times \vec{\Omega_0}=\vec{P}_{\rm ADM}$. Additionally, modify the
radius of the excision surface, $r_{\rm ex}$, to
drive $M_{\rm BH}$ to the desired value by applying (\cite{Buchman:2012dw})
\begin{equation}
\delta r_{\rm ex} = -r_{\rm ex} \frac{M_{\rm BH} - M_{\rm
    BH}^{*}}{M_{\rm BH}},
\end{equation}
where $M_{\rm BH}$ is the measured value in the initial data solve and
$M_{\rm BH}^{*}$ is the desired value.

\item Compute the spin of the BH by evaluating Eq.~\ref{eq:BhSpin}. Then
  modify the vector $\Omega^i_{\rm BH}$ in Eq.~\ref{eq:BhBoundary3} to drive
  the black hole spin to the target value,  by applying (\cite{Buchman:2012dw})
\begin{equation}
\delta \Omega_{\rm BH}^i = -\frac{\chi^i_{\rm BH}-\chi_{\rm
    BH}^{*i}}{4M}+\frac{M_{\rm BH}-M_{\rm BH}^{*}}{4M_{\rm
    BH}^2}\chi_{\rm BH}^i,
\end{equation}
where $\chi^i_{\rm BH}$ is the computed black hole spin, and $\chi_{\rm
    BH}^{*i}$ is the target spin.

\item
\label{it:omega}
 If desired, adjust the orbital angular frequency using
  Eq.~\ref{eq:OmegaDriver}, which, after exapnding the shift as 
$\beta^i=\beta^i_0 +
\vec{\Omega}\times\vec{r} + \dot{a}\vec{r}$, is a second-order
equation for $\Omega$.

\item Fix the Euler constant by evaluating the integral
\begin{equation}
M_{B}=\int \rho_0\Psi^6\gamma_ndV
\end{equation}
as a function of the Euler
constant $C$ (since $C$ is a function of $h$ and $h$ is a function of $\rho_0$), and using the secant method to drive the baryon mass to the
desired value.

\item Solve the elliptic equations for the velocity potential, $\phi$,
  and update using the same relaxation scheme as described above.

\item
\label{it:secondlast}
 Check whether all equations satisfied to the desired
  accuracy. If so, proceed to step~\ref{it:final}. Otherwise return to step~\ref{it:solve}.

\item 
\label{it:final}
Compute the truncation error for the current solution by
  examining the spectral coefficients of the metric variables. If the
  truncation error is too large (generally we use $10^{-9}$ as the criterion), adjust the number of grid points and
  return to step~\ref{it:1}. This adaptive refinement is based on the target truncation error and the measured convergence rate of the solution. See \cite{Szilagyi:2014fna} for a complete description
of this procedure.

\end{enumerate}

\section{Results}
\label{sec:BhNSResults}

%\subsection{bla}
%In this section we will demonstrate that we are able to construct robust initial data for generic bh-ns binaries with a spinning neutron star. To begin, we will focus on two axes of the mutli-dimensional parameter space - mass ratio and black-hole spin - while keeping the NS parameters fixed. In particular, we use a neutron star wtih ADM mass, $M_{\rm ADM}=1.4M_{\odot}$, equation of state $P=\kappa\rho_0^\Gamma\left(\Gamma=2.0, \kappa=101.45\right)$ and spin parameter, $\vec{\omega_{\rm NS}}=0.017\hat{z}$, corresponding to a dimensionless spin $\vec{\chi_{\rm NS}}\sim0.4\hat{z}$. The black hole is given a mass, $M_{\rm BH}=qM_{\rm ADM}$ and a dimensionless spin $\vec{\chi_{\rm BH}}=\chi\hat{z}$. In the first initial data sqeuence, we keep $q=7$ fixed, and vary $q=\{0,0.1,0.2,0.3,0.4,0.5,0.6,0.7,0.8,0.9,0.95\}$. In the second initial data sequence, we keep $\chi=0.9$ fixed and vary $q=\{2,3,4,5,6,7,8,9,10\}$. In each case, the BH and the NS are separated by a coordinate distance $D=7.44\left(M_{\rm BH}+M_{\rm ADM}\right)=7.44M$, and have an initial orbital angular velocity $M\Omega_{\rm orb}=0.0413$, with no additional infall velocity ($\dot{a}=0$). We will now focus on a subset of these initial data sets to demonstrate initial data convergence. In particular, the sets with $q=3$, $q=5$, $q=10$, $\chi=0$, $\chi=0.5$, $\chi=0.95$.



\subsection{Initial Data Set Parameters}
Let us begin by discussing the parameters for the initial data sets
that will be used in this work. We start from the six Bh-Ns
configurations described in Table 1 of \cite{Foucart:2013a}. All
binaries have a mass ratio $q=7$ and a black hole spin magnitude of
$\chi_{\rm BH}=0.9$. The neutron star in each binary has
an ADM mass of $M^{\rm Ns}_{\rm ADM}=1.4M_{\odot}$ - note that this is
distinct from the ADM energy, $E_{\rm ADM}$ of the full binary. In
practice, this means that it is fixed to have the same baryon mass as
that of an isolated, non-spinning neutron star with ADM mass
$1.4M_{\odot}$, with the same equation of state. We explore three
different Ns equations of state. All three are polytropic equations of
state ($P=\kappa\rho^\Gamma$) with $\Gamma=2$, but vary in terms of
$\kappa$, resulting in different neutron star compactnesses. The
values of $\kappa$ used are $84.28$, $92.12$, $101.45$, which result
in compactnesses of $0.170$, $0.156$, $0.144$, and radii of
approximately $12km$, $13km$, $14km$, respectively, for non-spinning
neutron stars with ADM Mass $1.4M_{\odot}$. We will use the notation
$R12$, $R13$ and $R14$ to refer to these configurations,
respectively. The $R12$ and $R13$ configurations are the onlys sets
with the black hole spin aligned with the orbital angular
momentum. For $R14$, we also consider precessing
black hole spins spanning angles of $20^{\circ}$, $40^{\circ}$ and
$60^{\circ}$ with the orbital angular momentum. The non-parallel part of the black hole spin is set
parallel to the $\hat{x}$ axis. In each case the initial separation
between the black hole and the neutron star is $D=7.44M$, where
$M=M_{\rm BH}+M_{\rm ADM}^{\rm NS}$ is the total mass of the
binary. The initial infall velocity parameter $\dot{a}_0$ is set to
$0$. The orbital angular velocity, $\Omega_0$, is the same as in \cite{Foucart:2013a} and is indicated in table~\ref{tab:FullBHNSParameters}.
The above constitutes 6 different configurations. 

For each of these 6
configurations, we choose an additional 6 configurations of neutron
star spins (thus 36 total configurations). In particular we choose
three directions - aligned with the orbital angular momentum,
anti-aligned with the orbital angular momentum, and completley in the
orbital plane, parallel to the $\hat{x}$ direction. For each of these
3 directions, we use a ``high-spin'' and a ``low-spin''. These are
typically $\chi_{NS}\sim 0.4$ and $\chi_{\rm NS}\sim 0.1$,
respectively. In our naming notation, we use a large arrow
($\Uparrow$)  for the ``high-spin'' configurations and a small arrow
($\uparrow$) for the ``low-spin'', with the direction of the arrow
indicating the direction of the Ns spin vector. The full parameters of
the initial data sets are summarized in Table
~\ref{tab:FullBHNSParameters}. In particular, the name of the run,
the black hole spin direction, the baryon mass of the neutron star, the orbital angular velocity of the system,
the approximate number of orbits before merger, the neutron star spin vector $\vec{\omega}$, and the measured dimensionless
spin of the neutron star. Note that the number of orbits is taken from
\cite{Foucart:2013a}, and does not account for the neutron star
spin. 

\begin{longtable}{l|c|c|c|c|c|c}
\centering
%\begin{tabular}{l|c|c|c|c|c|c}
\label{Tab:36Sets}
Name & $\Theta_{\rm BH}$ & $M^B_{\rm NS}$ & $M\Omega_{0}$ & $\sim N_{\rm orb}$ & $\vec{\omega}_{\rm NS}$ & $\vec{\chi}_{\rm NS}$  
\\\hline
{\tt R12i0$\uparrow$}&$0^\circ$ & 1.5212 & 0.0413 & 10.25 & $0.00667\hat{z}$ & $0.0995\hat{z}$ \\
{\tt R12i0$\Uparrow$}&$0^\circ$ & 1.5212 & 0.0413 & 10.25 & $0.0225\hat{z}$ & $0.4093\hat{z}$ \\
{\tt R12i0$\downarrow$}&$0^\circ$ & 1.5212 & 0.0413 & 10.25 & $-0.00667\hat{z}$& $-0.0895\hat{z}$\\
{\tt R12i0$\Downarrow$}&$0^\circ$ & 1.5212 & 0.0413 & 10.25 & $-0.0225\hat{z}$ & $-0.4030\hat{z}$ \\
{\tt R12i0$\rightarrow$}&$0^\circ$ & 1.5212 & 0.0413 & 10.25 & $0.00667\hat{x}$ & $0.0936\hat{x}$\\
{\tt R12i0$\Rightarrow$}&$0^\circ$ & 1.5212 & 0.0413 & 10.25 & $0.0225\hat{x}$ & $0.3989\hat{x}$ \\
\hline
{\tt R13i0$\uparrow$}&$0^\circ$ & 1.5128 & 0.0413  & 10.15 & $0.00555\hat{z}$ & $0.0997\hat{z}$ \\
{\tt R13i0$\Uparrow$}&$0^\circ$ & 1.5128 & 0.0413 & 10.15 & $0.019\hat{z}$ & $0.3911\hat{z}$ \\
{\tt R13i0$\downarrow$}&$0^\circ$ & 1.5128 & 0.0413 & 10.15 & $-0.00555\hat{z}$& $-0.0845\hat{z}$\\
{\tt R13i0$\Downarrow$}&$0^\circ$ & 1.5128 & 0.0413 & 10.15 & -$0.019\hat{z}$ & $-0.3793\hat{z}$ \\
{\tt R13i0$\rightarrow$}&$0^\circ$ & 1.5128 & 0.0413 & 10.15 & $0.00555\hat{x}$ & $0.0913\hat{x}$\\
{\tt R13i0$\Rightarrow$}&$0^\circ$ & 1.5128 & 0.0413 & 10.15 & $0.019\hat{x}$ & $0.3771\hat{x}$ \\
\hline
{\tt R14i0$\uparrow$}&$0^\circ$ & 1.5049 & 0.0413 & 9.85 & $0.005541\hat{z}$ & $0.1188\hat{z}$ \\
{\tt R14i0$\Uparrow$}&$0^\circ$ & 1.5049 & 0.0413 & 9.85 & $0.017\hat{z}$ & $0.4109\hat{z}$ \\
{\tt R14i0$\downarrow$}&$0^\circ$ & 1.5049 & 0.0413 & 9.85 & $-0.005541\hat{z}$& $-0.0965\hat{z}$\\
{\tt R14i0$\Downarrow$}&$0^\circ$ & 1.5049 & 0.0413 & 9.85 & -$0.017\hat{z}$ & $-0.3915\hat{z}$\\
{\tt R14i0$\rightarrow$}&$0^\circ$ & 1.5049 & 0.0413 & 9.85 & $0.005541\hat{x}$ & $0.1066\hat{x}$\\
{\tt R14i0$\Rightarrow$}&$0^\circ$ & 1.5049 & 0.0413 & 9.85 & $0.017\hat{x}$ & $0.3907\hat{x}$ \\
\hline
{\tt R14i20$\uparrow$}&$20^\circ$ & 1.5049 & 0.0412 & 9 & $0.005541\hat{z}$ & $0.1188\hat{z}$ \\
{\tt R14i20$\Uparrow$}&$20^\circ$ & 1.5049 & 0.0412 & 9 & $0.017\hat{z}$ & $0.4110\hat{z}$ \\
{\tt R14i20$\downarrow$}&$20^\circ$ & 1.5049 & 0.0412 & 9 &  $-0.005541\hat{z}$& $-0.0964\hat{z}$\\
{\tt R14i20$\Downarrow$}&$20^\circ$ & 1.5049 & 0.0412 & 9 & -$0.017\hat{z}$ & $-0.3915\hat{z}$ \\
{\tt R14i20$\rightarrow$}&$20^\circ$ & 1.5049 & 0.0412 & 9 & $0.005541\hat{x}$ & $0.1064\hat{x}$\\
{\tt R14i20$\Rightarrow$}&$20^\circ$ & 1.5049 & 0.0412 & 9 & $0.017\hat{x}$ & $0.3905\hat{x}$ \\
\hline
{\tt R14i40$\uparrow$}&$40^\circ$ & 1.5049 & 0.0412 & 8.5 &  $0.005541\hat{z}$ & $0.1193\hat{z}$ \\
{\tt R14i40$\Uparrow$}&$40^\circ$ & 1.5049 & 0.0412 & 8.5 & $0.017\hat{z}$ & $0.4117\hat{z}$ \\
{\tt R14i40$\downarrow$}&$40^\circ$ & 1.5049 & 0.0412 & 8.5 & $-0.005541\hat{z}$& $-0.0961\hat{z}$\\
{\tt R14i40$\Downarrow$}&$40^\circ$ & 1.5049 & 0.0412 & 8.5 & -$0.017\hat{z}$ & $-0.3908\hat{z}$ \\
{\tt R14i40$\rightarrow$}&$40^\circ$ & 1.5049 & 0.0412 & 8.5 & $0.005541\hat{x}$ & $0.1064\hat{x}$\\
{\tt R14i40$\Rightarrow$}&$40^\circ$ & 1.5049 & 0.0412 & 8.5 & $0.017\hat{x}$ & $0.3905\hat{x}$ \\
\hline
{\tt R14i60$\uparrow$}&$60^\circ$ & 1.5049 & 0.0415 & 7 &  $0.005541\hat{z}$ & $0.1200\hat{z}$ \\
{\tt R14i60$\Uparrow$}&$60^\circ$ & 1.5049 & 0.0415 & 7 & $0.017\hat{z}$ & $0.4132\hat{z}$ \\
{\tt R14i60$\downarrow$}&$60^\circ$ & 1.5049 & 0.0415 & 7 & $-0.005541\hat{z}$& $-0.0954\hat{z}$\\
{\tt R14i60$\Downarrow$}&$60^\circ$ & 1.5049 & 0.0415 & 7 & -$0.017\hat{z}$ & $-0.3898\hat{z}$ \\
{\tt R14i60$\rightarrow$}&$60^\circ$ & 1.5049 & 0.0415 & 7 & $0.005541\hat{x}$ & $0.1061\hat{x}$\\
{\tt R14i60$\Rightarrow$}&$60^\circ$ & 1.5049 & 0.0415 & 7 & $0.017\hat{x}$ & $0.3903\hat{x}$ \\
%\end{tabular}
\caption[Initial data set parameters for series of 36 Bh-Ns initial
data sets.]{\label{tab:FullBHNSParameters}Full set of parameters of
  the 36 sets of initial data constructed here. $\theta_{\rm BH}$ is
  the angle between the black hole spin and the orbital angular
  momentum, $M^B_{\rm NS}$ is the baryon mass of the neutron star,
  $M\Omega_0$ is the orbital frequency of the binary, $\sim N_{\rm
    orb}$ is the approximate number of orbits before merger,
  $\vec{\omega}_{\rm NS}$ is the spin vector of the neutron star as
  defined in Eq.~\ref{eq:RotationTerm}, and $\vec{\chi}_{\rm NS}$ is
  the measured dimensionless spin of the neutron star.}
\end{longtable}

Although we do not perform evolutions of any of these initial data sets, in principle these initial data sets could be easily used to
extend the work of \cite{Foucart:2013a}. For example, it was found
that the size of the neutron star, varying from $12km$ to $14km$ can
greatly affect the onset of neutron star disruption, and thus the disk
produced and the electromagnetic signal that would be produced. Given
such sensitivity to the Ns parameters, it is natural to think that
sensitivity to the Ns spin should be tested. If the material is less
strongly bound to the star because of centrifugal forces arising from
the spin, then the stars should disrupt earlier.

\subsection{Convergence of the Initial Data Solver}
To assess the convergence of the initial data solver we will begin by looking at the convergence of the iterative part of the solver. That is, the convergence of steps 1-12 in the iterative procedure
described above. We will first focus on one particular initial data
set of the 36 in Tab~\ref{Tab:36Sets} - namely the {\tt
  R14i60$\Uparrow$} initial data set. The results we present for {\tt
  R14i60$\Uparrow$} are representative for all of the 36 sets considered.

We begin by looking at the convergence of the Euler constant, $C$. In figure~\ref{Fig:EulerConv} we plot the absolute difference in $C$ between neighbouring iterations for the eight
different resolutions used in the initial data solve. %\red{Expand explanation}In our notation, {\tt R1} is the first (lowest) resolution, {\tt R2} is the second, etc. 
In the figure we see that at a given resolution these differences
decrease exponentially with iteration as expected for the relaxation
scheme employed (cf. Eq.~\ref{eq:Relaxation}). Meanwhile the differences also decrease with increasing resolution. This all shows that the initial data solver is working as it should. Note that although we've only shown
this plot for the {\tt R14i60$\Uparrow$} inital data set, we find similar results for all the other initial data sets we consider.


\begin{figure}
\includegraphics[width=0.95\columnwidth]{chap4/EulerConv}
\caption[Convergence of the Euler constant for the {\tt
  R1460$\Uparrow$}]{\label{Fig:EulerConv}
Absolute difference between neighbouring iterations of the Euler
constant for the {\tt R14i60$\Uparrow$} initial data set. The eight
curves are the eight different resolutions
used in the initial data solve.}
\end{figure}

Next, we will look at the properties of the black hole to verify that they converge as expected in the presence of a spinning neutron star. Again, we will focus on the {\tt R14i60$\Uparrow$} run, and look at the black hole spin parameter $a=J/M=M\chi$, which is controlled by the parameter $\Omega_j^{\rm BH}$ in Eq.~\ref{eq:BhBoundary3}. The target values are
\begin{eqnarray}
a&=&\chi_{\rm BH}M_{\rm BH}\\
&=&q\chi_{\rm BH}M_{\rm NS}^{\rm ADM}\\
a_x&=&q\chi_{\rm BH}(\sin60^{\circ})M_{\rm NS}^{\rm ADM}\\
a_z&=&q\chi_{\rm BH}(\cos60^{\circ})M_{\rm NS}^{\rm ADM}
\end{eqnarray}
We also consider the convergence of the irreducible mass of the black hole, $M_{\rm irr}$. It is controlled by the size of the excision surface, and satisfies the relation
\begin{eqnarray}
M_{\rm BH}^{\rm irr}&=&\sqrt{\frac{A_{\rm AH}}{16\pi}}\\
(M_{\rm BH}^{\rm ADM})^2\left((M_{\rm BH}^{\rm irr})^2-\frac{a^2}{4}\right)&=&(M_{\rm BH}^{\rm irr})^4
\end{eqnarray}

In figure~\ref{Fig:BHSpinConv} we plot the fractional difference between the measured black hole spin and the target value, for both the x and z directions, as well as the fractional difference between
the measured black hole irreducible mass and the target value. The
difference is plotted as a funciton of iteration, for four different
resolutions. Note that the previous plot had eight different
resolutions, while this one only has four. This is because the black
hole spin is only computed for the first four resolutions - afterwards
it is no longer solved adjusted (cf. Step~\ref{it:toplevelparamsolve}) We note that in general we see a decrease in this difference with iteration, especially at the first resolution, therefore showing that the iterative solver is correctly driving the the black hole properties to the target values.
We also note that this difference decrease with resolution, and that
we are able to achieve an accuracy of about $10^{-5}$ in the BH spin
and mass. \red{Can we understand this sawtooth pattern?}

\begin{figure}
\includegraphics[width=0.95\columnwidth]{chap4/BHSpinConv}
\caption[Convergence of black hole spin and
mass.]{\label{Fig:BHSpinConv}Plotted is the fractional difference from the target black hole spin and black hole mass in the {\tt R14i60$\Uparrow$} initial data set as a function of iteration. The four colours represent 
the four different resolutions at which the black hole spin is measured. The solid curves are for the $\hat{x}$ component of the BH spin, the dashed curves for the $\hat{z}$ component of the BH spin, and
the dotted curves are for the irreducible mass of the BH.}
\end{figure}

Having now established the convergence of the iterative procedure, we
turn now to establish the convergence with resolution of the global
properties of the solution, continuing to focus on the {\tt R14i60$\Uparrow$} ID set.We begin by looking at the Hamiltonian and momentum constraints for the {\tt R14i60$\Uparrow$} initial data set.
The Hamiltonian constraint is computed as
\begin{equation}
H=||\frac{R_{\Psi}}{8\Psi^5}||
\end{equation}
where $R_{\Psi}$ is the residual of Eq.~\ref{eq:XCTS-ConformalFactor}
and $||.||$ represents the $L2$ norm over the computationalT
domain. Similarly, the momentum constraint is computed as
\begin{equation}
M = ||\frac{R_{\beta}}{2\alpha\Psi^4}||
\end{equation}
where $R_{\beta}$ is the residual of Eq.~\ref{eq:XCTS-Shift}.
The constraints for this ID set are shown in figure~\ref{fig:HamMom}. We find exponential convergence in the constraints, as expected for spectral methods. The most notable feature
is that at the second resolution the constraints are higher than would be expected
from the trend, otherwise. This is because the neutron star spin is
only "turned on" during the second resolution (cf. step~\ref{it:1}), and so an additional amount of
solver time is needed to adjust to this change.
\begin{figure}\includegraphics[width=0.95\columnwidth]{chap4/HamMom}
\caption[Hamiltonian and momentum constraints of the {\tt R14i60$\Uparrow$} ID set]{\label{fig:HamMom} The Hamiltonian and momentum constraints for the {\tt R14i60$\Uparrow$} initial data set
as a function of resolution. We find exponential convergence in both.}
\end{figure}

\red{Improve wording of this paragraph}Finally, we look at the properties of the neutron star. As noted in Eq.~\ref{eq:NSSurf},
the neutron star surface is expressed as a sum of spherical
harmonics.
To evaluate the convergence of the surface location, we define the
quantity,
\begin{equation}
\label{eq:NSSurf2}
\Delta c(N)=\sqrt{\sum_{l,m}^{l_{\rm max},m_{\rm max}}\left(c_{lm}(N)-c^{*}_{lm}\right)^2}
\end{equation}
where $N$ represents the current resolution, and $*$ represents the
highest resolution. This quantity is plotted in
figure~\ref{fig:SpinDiff}. Similar to the black hole surface, the
neutron star surface is only computed for the first four resolutions,
and so we have three data points shown. We find exponential
convergence in this quantity. We also look at the convergence of the
neutron star spin $\chi_{\rm NS}$ measured at each resolution. In
figure~\ref{fig:SpinDiff}, we plot the fractional difference in
$\chi_{\rm NS}$ between neighbouring resolutions. That is, we plot
$
\Delta\chi=|\frac{\chi(N+1)-\chi(N)}{\chi(N)}|.
$
 We find apparent
exponential convergence, although there are two disinctly different
slopes in the data, with the slope becoming more shallow at high
resolution. This is likely because we have stopped measuring the Ns
surface at this point, i.e., after the fourth resolution. However, we are able to measure the spin to an
accuracy of about $10^{-6}$. 
 We have omitted the first data point in this curve
because the neutron star spin is not turned on in the first
resolution, so by construction the first data point is artificially large.

\begin{figure}
\includegraphics[width=0.95\columnwidth]{chap4/SpinDiff}
\caption[Neutron star surface and spin accuracy.]{\label{fig:SpinDiff}
\red{Re-visit this plot}
  Accuracy of neutron star properties. Plotted are, as a function of resolution the accuracy of the computation of the neutron star
  surface, $\Delta c$, as defined in Eq.~\ref{eq:NSSurf2} (solid black
  curve), and the fractional difference between neighbouring
  resolutions in the measured neutron star spin $\chi_{\rm NS}$ (dashed red
  curve). The first data point is omitted for the latter curve since it is artifically high by construction - the Ns spin is set to zero during the first resolution. This is for the {\tt R14i60$\Uparrow$} ID set.}
\end{figure}

The above data all show that we have established the convergence of
our initial data solver, by showing exponential convergence of the
iterative solver, the black hole properties, neutron star properties,
and the constraints.

\subsection{Other Initial Data Sequences}
In addition to the parameters discussed above, we consider several other sequences
of Bh-Ns initial data sets to verify the robustness of the solutions across
the whole parameter space. First, we consider a sequence of where we vary the neutron star spin from
$\chi_{\rm NS}=0$ to $\chi_{\rm NS}\sim0.7$, keeping it aligned with the orbital angular momentum. In these initial data sets, the other binary parameters are the same
as in the {\tt R14i0} runs. Namely, the neutron star mass, equation of state, black hole mass, black hole spin, initial separation and orbital angular frequency.

Next, we consider a sequence of runs where we vary the black hole spin
from $\chi_{\rm BH}=0$ to $\chi_{\rm BH}=0.99$, while keeping the other binary parameters as in the {\tt R14i0$\Uparrow$} run.

Finally, we consider a sequence of runs where we vary the mass ratio
from $q=2$ to $q=10$. In these runs the other binary parameters are
the same as in the {\tt R14i0$\Uparrow$} run. In particular, the
orbital parameters $M\Omega$ and $D/M$ are constant. While not the
most accurate way of choosing these parameters, as it is only correct to Newtonian order, it is a very simple method
that is sufficient for our purposes, as it will only induce a small orbital eccentriccity. To assess the accuracy of this approximation, we use the post-Newtonian expansion in Eq. 228 of \cite{Blanchet2006}:
\begin{equation}
\Omega^2=\frac{GM}{r^3}\left(1+(-3+\nu)\gamma+\left(6+\frac{41}{4}\nu+\nu^2\right)\gamma^2+...\right)
\end{equation}
where $\nu$ is the symmetric mass ratio, $\nu=q/(1+q)^2$, and $\gamma=GM/Dc^2$. Keeping $D$ and $M$ constant, the quantity $M\Omega$ varies by approximately 3\% in the mass ratio range we consider.
Thus, while this approximation will induce a small eccentricity of a similar order, it's not a large enough difference to invalidate our results.

Figure~\ref{fig:3dparam} summarizes all the initial data sets along the axes of $\chi_{\rm NS}$, $\chi_{\rm BH}$ and $q$.

\begin{figure}
\includegraphics[width=0.95\columnwidth]{chap4/3dparam.png}
\caption[3d parameter space plot of Bh-Ns initial data sets.]{\label{fig:3dparam}
Parameter space exploration. Starting from {\tt
  R14i60$\Uparrow$} (large red circle) we vary (i) the Bh spin
$\chi_{\rm BH}$, (ii) the Ns spin $\chi_{\rm NS}$ and (iii) the black
hole mass $M_{\rm BH}$, indicated by the mass-ration $q$.}
\end{figure}

We begin by
varying the neutron star spin from $\chi=0$ ($\omega=0$) to $\chi\sim0.7$,
corresponding to $\omega\sim0.22$. The parameters in this
sequence are otherwise the same as in the {\tt R14i0} data sets, and the neutron star spin is kept aligned with the orbital
angular momentum. To demonstrate convergence, we begin by plotting the
Hamiltonian and momentum constraints for a subset of the generated
initial data sets, with $\chi\sim{0.1,0.3,0.5,0.7}$. This is shown in
Fig.~\ref{fig:OmegaSeqHamMom}. We note that in general we see
exponential convergence, as expected, but there are a few features
worth discussing in the data.
As discussed earlier, the constraints at the second resolution are higher than expected because the spin is "turned on" at the second resolution.
The size of this feature at is a monotonically increasing function of spin, as we might
expect, as the solver has a more difficult time adjusting to a larger spin being abruptly turned on. We also note that at high resolution, in
the $\chi\sim 0.7$ curve, we lose exponential convergence and the
curves flatten out around $10^{-6}$.  \red{Plot $\delta~C$ for
  different $\chi_{\rm NS}$.}This is likely a sign that the accuracy of the
solver is becoming limited, likely by approximations that go into the
solver. $\chi\sim 0.7$ is around the maximum theoretical neutron star
spin, so such difficulties are expected.

\begin{figure}
\includegraphics[width=0.95\columnwidth]{chap4/OmegaSeqHamMom}
\caption[Constraints for the $\chi_{\rm NS}$ sequence.]{\label{fig:OmegaSeqHamMom} Hamiltonian (solid curves) and
momentum (dotted curves) constraints for four different neutron star spins.}
\end{figure}

In Fig.~\ref{fig:ChiVOmega} we plot the
measured neutron star spin $\chi_{\rm NS}$ as a function of the code
parameter $\omega$ for the full sequence. As expected, we find a linear
relationship at low $\omega$, but the relationship becomes non-linear
at higher $\omega$, as the neutron star's size, and thus moment of
inertia, becomes an appreciable function of spin. We find that the
solver breaks down around $\chi_{\rm NS}\sim 0.7$ which is around the
theoretical maximum spin for $\Gamma=2$ polytropes. In
Fig.~\ref{fig:ChiVOmega} we have also plotted, in addition to this
Bh-Ns curve, a curve from Ns-Ns initial data sets. These data use
different Ns parameters, with mass $M_{\rm ADM}=1.64M_{\odot}$ and
equation of state parameter $\kappa=123.6$. However, the curves remain
very close to each other in shape. Thus we see the neutron star spin
code seems to be working the same for the code described in this
chapter, as well as the code described in Ch. \red{2}.

\begin{figure}
\includegraphics[width=0.95\columnwidth]{chap4/chiVOmega}
\caption[$\chi_{\rm NS}$ as a function of $\omega_{\rm NS}$ for bh-ns
binaries]
{\label{fig:ChiVOmega}
Neutron star spin $\chi$ as a function of neutron star spin parameter
$\omega$ for a sequence of initial data sets. The black hole spin is
constant at $\chi=0.9$ and the mass ratio is $q=7$. The dashed red curve is
from Ns-Ns binaries, with somewhat different neutron star parameters.}
\end{figure}

After the sequence in $\chi_{\rm NS}$, we next look at a sequence in
$\chi_{\rm BH}$. In partiucIar, we vary the black hole spin from
$\chi_{\rm BH}=0$ to $\chi_{\rm BH}=0.99$, keeping it aligned with the orbital angular
momentum.
 The other binary parameters are kept the same
as in the ${\tt R14i0\Uparrow}$ initial data set; that is $\chi_{\rm
  NS}\sim0.4$ and $q=7$. In Figs.~\ref{fig:chiSeqHam} we plot the Hamiltonian and momentum constraints,
respectively, for this sequence. We find exponential convergence in
all cases. It is interesting to note that the constraints seem to be
lowest at the highest black hole spins, $\chi_{\rm BH}=0.95$ and
$\chi_{\rm BH}=0.99$, while one might expect these to be the most
challenging cases.

\begin{figure}
\includegraphics[width=0.95\columnwidth]{chap4/chiSeqHam}
\caption[Hamiltonian and momentum for the sequence in $\chi_{\rm BH}$.]{\label{fig:chiSeqHam}Hamiltonian (top panel) and momentum (bottom panel) constraints versus
  resolution for our sequence of binaries where the black-hole spin is
  varied from $\chi_{\rm BH}=0$ to $\chi_{\rm BH}=0.99$. The neutron star spin is constant at $\chi_{\rm NS}\sim 0.4$ and the mass ratio is $q=7$. We find exponential convergence in all cases.}
\end{figure}

Since in this work we are largely concerned with neutron star spin, it
is interesting to consider how the measured neutron star spin,
$\chi_{\rm NS}$ couples to other binary parameters. To lowest order,
it should depend only on $\omega_{\rm NS}$, but in practice it may
also depend on the parameters of the black hole or of the orbit. For this sequence in
$\chi_{\rm BH}$, we plot $\chi_{\rm NS}$ as a function of $\chi_{\rm
  BH}$ in figure~\ref{fig:chichi}. $\chi_{\rm NS}$ is nearly constant,
dropping by less than 1\% between $\chi_{\rm BH}=0$ and $\chi_{\rm BH}=0.99$.

\begin{figure}
\includegraphics[width=0.95\columnwidth]{chap4/chichi}
\caption[Measured neutron star spin plotted as a function of black hole spin.]{\label{fig:chichi}Neutron star spin $\chi_{\rm NS}$ as a function of black hole spin $\chi_{\rm BH}$ for this sequence. We notice a small downward linear trend.}
\end{figure}




Finally, we consider a sequence where we vary the mass ratio from
$q=2$ to $q=10$. In this sequence we keep the other binary parameters
the same as in the ${\tt R14i0\Uparrow}$ initial data set and we keep the
orbtial parameters $M\Omega$ and $D/M$ constant. As discussed above,
we expect that this will induce a eccentricity of a few percent. To
assess convergence, we plot the Hamiltonian and momentum constraints
for this sequence in Fig.~\ref{fig:qSeqHam}. We find exponential convergence in all
cases. Interestingly, no clear pattern in $q$-space emerges.

\begin{figure}
\includegraphics[width=0.95\columnwidth]{chap4/qSeqHam}
\caption[Hamiltonian and momentum constraint for the $q$ sequence.]{\label{fig:qSeqHam}Hamiltonian (top panel) and momentum (bottom panel) constraints versus resolution for our sequence of binaries where the mass ratio is varied from $q=2$ up to $q=10$. The neutron star spin is constant at $\chi_{\rm NS}\sim 0.4$ and the black hole spin is $\chi_{\rm BH}=0.9$. We find exponential convergence, as expected, in all cases.}
\end{figure}

As with the previous sequence, it is interesting to look at the
properties of $\chi_{\rm NS}$ as a function of $q$. This is shown in
Fig~\ref{fig:qchi}. Although there is not a great amount of variation,
apart from $q=2$, there is a clear trend of
$\chi_{\rm NS}$ decreasing with $q$. Again, howveer, the effect is quite small and we do not seek to explain it.
\begin{figure}
\includegraphics[width=0.95\columnwidth]{chap4/qchi}
\caption[$\chi_{\rm NS}$ as a function of mass ratio
$q$.]{\label{fig:qchi}Neutron star spin $\chi_{\rm
    NS}$ as a function of mass ratio $q$ for this sequence. We notice
  a small downward trend for $q \geq 3$.}
\end{figure}



%\section{Comparison with Black-Hole Binaries}
%Comparison

\section{Conclusion}
\label{sec:Conc}
We have presented code to create initial data for Bh-Ns binaries where the neutron stars have arbitrary spins. To our knowledge, this is the first time initial data has been created for Bh-Ns binaries with spinning neutron stars. We used the same formalism develped by~\cite{Tichy:2011gw} and used in~\cite{Tacik:2015tja} to create
initial data for Ns-Ns systems with arbitrary spins. Two new numerical tricks were found to be necessary to get convergent initial data - setting a maximum radius out to which to apply $W^i=\epsilon^{ijk}\omega^jr^k$, and only "turning on" the neutron star spin after the first resolution of the initial data solver has completed. We created convergent initial data sets across a large portion of the Bh-Ns binary parameter space. We created 36 initial data sets using the results of~\cite{Foucart:2013a} as a starting point. These initial data sets spanned three different equations of state (all $\Gamma=2$ polytropes), two different neutron star spin magnitudes, three different neutron star spin orientations, and four different black hole spin orientations. We also created three sequences of inital data to help fill out the parameter space - a sequence in neutron star spin ranging from $\chi_{\rm NS}=0$ to $\chi_{\rm NS}\sim0.7$ (near the maximum theoretical spin for neutron stars), a sequence in black hole spin ranging from $\chi_{\rm BH}=0$ to $\chi_{\rm BH}=0.99$ and a sequence in mass ratio ranging from $q=2$ to $q=10$.

Future research will involve running evolutions of these, or similar,
initial data sets. Some of the 36 initial data sets could be used to
investigate how neutron star spin affects tidal disruption of the star
by the black hole, and how it affects the disk that is formed. The
phase evolution can also be examined and compared to Post-Newtonian or
other analytic predictions such as Effective-One-Body (EOB). In long
and accurate simulations, the coupling of the neutron star spin to the
dynamic tides (\cite{Hinderer:2016eia}) could become important in this kind of comparison. Another interesting route to explore could be to see how large of a disk can be produced in Bh-Ns simulations. \cite{Lovelace:2013vma} create a very large disk with a black hole spin of $\chi=0.97$ and mass ratio $q=3$ - a highly spinning neutron star with less strongly bound material may be able to increase this disk mass even further.

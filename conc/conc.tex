\chapter{Conclusions \& Future Work}
\label{chap:conc}

\section{Conclusions}
With the recent direct detection of gravitational waves, it has never been more exciting to be in the numerical relativity community. Based on the preliminary event rate predictions, the first detection of gravitational waves from a neutron star cannot be far away, and with that comes the exciting possibility of the detection of an electromagnetic counterpart.

In this thesis we have largely focused on spinning neutron stars in compact object binaries. In chapter~\ref{chap:bns} we introduced our code and methodology for creating initial data for binary neutron star systems with arbitrary spins, and showed that we are able to created convergent, constraint-satisfying initial data. We introduced a novel method to directly measure the spin of the neutron stars using quasilocal approximate Killing vectors, and we showed that this method is accurate and robust. We also evolved three binary configurations, including one highly-spinning precessing system. We showed that the properties of these systems agree remarkably well with Post-Newtonian predictions, even without account for the neutron star tidal terms. We also showed that we are able to control the eccentricity of the systems to an accuracy of $\sim 0.1\%$. Upon evolving the systems, we found large density oscillations in the neutron stars. We determined that these were due to an excited quasi-normal mode, due to imperfect initial data. As discussed in the appendix of~\ref{chap:bns}, however, we note that these oscillations were present because of an error that was present in the code. Upon fixing that error, the density oscillations drop by a factor of $\sim 40$, and that we are able to construct initial data with much higher spins than previously. The work done in~\ref{chap:bns} leaves open maybe possible directions for future studies. At the time of writing~\ref{chap:bns} the {\tt SpEC} code had a difficult time merging binary neutron stars, and so we were only able to simulate orbits of the late inspiral. It remains an open, and interesting, as to how high NS spin affects the dynamics of the merger. Given the results of the appendix, future studies should be able to use even higher spinning stars. One can then investigate how well Post-Newtonian match with the numerical relativity simulations at even higher spins. It is also interesting to ask how the spin interacts with neutron star compactness and binary mass ratio in the dynamics of the inspiral and merger.

In chapter~\ref{chap:bhns} we extended the code of chapter~\ref{chap:bns} to create initial data for black hole--neutron star (BHNS) binaries. To show the robustness of the code, we create many different initial data sets across the BHNS paramter space. We vary the neutron star spin from $\chi_{\rm NS}=0$ to $\chi_{\rm NS}\sim 0.7$, near the neutron star mass-shedding limit, the black hole spin from $\chi_{\rm BH}=0$ to $\chi_{\rm BH}=0.99$, and the mass ratio from $q=2$ to $q=10$. We also vary the directions of the neutron star spin and black hole spin, and consider three different compactnesses of the neutron star. An obvious extension of this work would be to evolve and merge some of these initial data sets. We've created 36 initial data sets that vary from already published data sets~\citep{Foucart:2013a}, and so this would facilitate the comparison quite well. One could also use spinning NS in BHNS systems to try to explore the maximum mass of accretion disks - adding spin on the neutron star should make the material less strongly bound and increase the disk mass further.

Finally, in chapter~\ref{chap:jr} we studied the paramter space dependence of spurious, "junk", radiation in binary black hole (BBH) simulations. To measure the amount of junk radiation present we introduced three diagnostics - the energy in the pulse of junk radiation, and the mass increase and spin decrease transient quantities. We considered how these depend on the spins and initial separations of the black holes, and we compared conformally flat (CF) initial data with superposed Kerr-Schild (SKS) initial data. In terms of the energy present, we found that it does not depend signficantly on spin, but found power law relations with the initial separations. The energy present in the CF data was larger by a factor of $2-3$. We were unable to directly quantify the transient quantities for SKS initial data beacuse of their small magnitude and their lack of convergence. For the CF data we found weak dependence on initial separation, but a strong power law dependence on spin for the mass increase, and an exponential dependence on spin for the spin decrease. There are several directions in which one could extend this research. The same sort of analysis could be performed using a different type of initial data, such as those adding tidal effects or outgoing gravitational radiation. One could also extend these methods further across the BBH paramter space - going to higher black hole spins, non-equal mass ratios, non-aligned spins, or further/closer separations.
